\documentclass[twocolumn]{rbef}
\usepackage{lipsum}

\usepackage{bbm}
\usepackage{subfig}

\newcommand{\1}{\mathbbm{1}}
\newcommand{\s}{\mathcal{S}}
\newcommand{\T}{\mathcal{T}}
\newcommand{\A}{\mathcal{A}}
\newcommand{\ket}{\rangle}
\newcommand{\bra}{\langle}

\newtheorem{defi}{Definição}
\newtheorem{theorem}{Teorema}
\newtheorem{acknowledgement}[theorem]{Acknowledgement}
\newtheorem{algorithm}[theorem]{Algorithm}
\newtheorem{axiom}[theorem]{Axiom}
\newtheorem{claim}[theorem]{Claim}
\newtheorem{conclusion}[theorem]{Conclusion}
\newtheorem{condition}[theorem]{Condition}
\newtheorem{conjecture}[theorem]{Conjecture}
\newtheorem{corollary}[theorem]{Corollary}
\newtheorem{criterion}[theorem]{Criterion}
\newtheorem{definition}[theorem]{Definition}
\newtheorem{example}[theorem]{Example}
\newtheorem{exercise}[theorem]{Exercise}
\newtheorem{lemma}[theorem]{Lemma}
\newtheorem{notation}[theorem]{Notation}
\newtheorem{problem}[theorem]{Problem}
\newtheorem{proposition}[theorem]{Proposition}
\newtheorem{remark}[theorem]{Remark}
\newtheorem{solution}[theorem]{Solution}
\newtheorem{summary}[theorem]{Summary}
\newenvironment{proof}[1][Proof]{\noindent\textbf{#1.} }{\ \rule{0.5em}{0.5em}}

\titulocabecalho{Nostradamus: plataforma de aprendizado de máquina como um serviço para tratamente, análise, visualização e previsão de séries temporais providas pelo usuário}
\autorcabecalho{J. T. Anchante and A. R. A. Grégio}

\numeracao{01}
\volume{01}
\numero{01}
\ano{2019}
\doi{http://dsbd.leg.ufpr.br/tcc}
% \tipodeartigo{TCC DSBD}
\tipodeartigo{Especialização em Data Science \& Big Data}
% \addtocounter{page}{566} %% \setcounter produces extra white page!!! use ===\addtocounter===

\author[1]{Jayme T. Anchante}

\affil[1]{Departamento de Física, Universidade Regional do Cariri
  Av. Leão Sampaio 107, Triângulo, 63041-082, Juazeiro do Norte, CE,
  Brasil\thanks{\href{emailto:jayme.anchante@disroot.org}{jayme.anchante@disroot.org}}
}

\author[2]{André R. A. Grégio}

\affil[2]{Departamente de Informática, Universidade Federal do Paraná
  Rua Cel. Francisco Heráclito dos Santos, 100 – Centro Politécnico, Jardim das Américas, 81531-980, Curitiba,
  PR, Brasil\thanks{\href{emailto:gregio@ufpr.br}{gregio@ufpr.br}}
}

\titulo{Nostradamus: plataforma de aprendizado de máquina como um serviço para tratamente, análise, visualização e previsão de séries temporais providas pelo usuário}

\subtitulo{Nostradamus: machine learning as a service platform for treatment, analysis, visualization and prediction of time series provided by the user}

% -----------------------------------------------------------------------

\begin{document}

\begin{primeirapagina}

  % \begin{center}
  %   \vspace{-12pt} \small{Recebido em xxx. Aceito em xxx}
  % \end{center}

  \begin{abstract}
    Lorem ipsum
    sit amet
    \palavraschave{aprendizado de máquina, automl,
      séries temporais, aprendizado profundo}

  \end{abstract}

  \begin{otherlanguage}{english}


    \begin{abstract}
      Lorem ipsum
      sit amet
      \keywords{machine learning, automl,
        time series, deep learning}

    \end{abstract}
  \end{otherlanguage}

\end{primeirapagina}
\saythanks

\section{Introdução} \label{introduction}

Série temporal são dados pontuais ordenados temporalmente. Apesar do tempo ser contínuo, normalmente uma série temporal envolve dados discretos, tomados em sequência de períodos igualmente espaçados e sucessivos de tempo. Além da análise de séries temporais, ou seja, métodos para extrair estatísticas úteis e outras características dos dados, outro tema bastante relevante e objeto de estudos do presente trabalho é a predição de séries temporais, processo que envolve o uso de modelos que predigam valores futuros com base em valores passados observados.

A previsão de séries temporais é um tema bastante relevante e comum na vida de todos. Indivíduos verificam a previsão do tempo do dia seguinte para saber como se agasalhar ou se devem levar um guarda chuva, lemos no jornal a previsão da inflação para os próximos meses ou a cotação do dólar futuro, planejamos quanto iremos gastar por mês ao longo do ano ou até dos anos seguintes. Governos precisam prever os gastos do ano seguinte para ajustar o provimento do orçamento, prever o crescimento dos demais países para avaliar um acordo de comércio e o Banco Central deve prever a inflação dos próximos meses para fazer um ajuste na taxa de juros básica da economia. Empresas devem prever o demanda por seus produtos no varejo para que as lojas não fiquem desabastecidas, prever a sua receita nos próximos anos para avaliar um potencial investimento e exportadoras prevêem o dólar para que não incorram em prejuízos cambiais.

Séries temporais diferenciam-se de estudos de seção cruzada, pois estes não possuem nenhuma ordenação no tempo, como por exemplo o efeito da educação no salário dos indivíduos, em que os dados de diferentes pessoas podem colocados em qualquer ordem sem que haja prejuízo para a modelagem, ou ainda os dados são reflexo de um certo período de tempo, como por exemplo um censo decenal ou uma pesquis anual. Diferenciam-se ainda de dados espaciais, pois estes possuem uma dependência geográfica, por exemplo o preço de imóveis depende da localização assim como de variáveis intrínsecas dos imóveis.

A análise de séries temporais pode ser dividia entre dois domínios: de frequência e de tempo. O último envolve o uso de funções matemáticas  ou sinais com respeito a frequência, ou seja, o quanto do sinal fica entre cada banda de frequência ao longo de uma série de frequências, algumas das transformações matemáticas mais utilizadas são as transformações de Fourier, de Laplace, Z e de Ondas, com aplicações em ondas sonoras, circuitos eletrônicos, processamento de sinal digital e compressão de dados. O domínio do tempo envolve o uso de funções matemáticas ou sinais com respeito ao tempo, que serão tratadas ao longo do presente trabalho.

Ainda, as técnicas de séries temporais podem ser divididas entre paramétricas e não paramétricas. As paramétricas assumem que os dados possuem um estrutura que pode ser descrita por um certo número finito (e normalmente pequeno) de parâmetros, que serão estimados para descrever o processo gerador dos dados, abordagem proposta pelo presente estudo. As técnicas não paramétricas tentam estimar diretamente os diferentes momentos dos dados (como média, variância e covariância) sem que seja assumido que o processo gerador possua qualquer estrutura em particular.

Por fim, a análise pode ser feita de forma univariada ou multivariada. O caso univariado trata de apenas uma série temporal, única e exclusivamente, enquanto que no caso multivariado, uma ou mais variáveis que podem ter uma relação de dependência são analisadas simultaneamente. Trataremos apenas do caso univariado.


O presente trabalho está organizado como se segue: analisaremos a forma como a estatística tradicional e como a de ciência de dados modelam séries temporais nas seções 1 e 2, respectivamente, além de uma introdução de cada abordagem, cada seção também disporá de duas subseções que tratarão dos principais modelos utilizados assim como as medidas de sucesso de cada abordagem. A seção 3 mostrará um benchmark das duas abordagens predizendo bases de dados amplamente utilizadas trabalhos acadêmicos ou medidas bastante populares.

\section{Abordagem estatística clássica de séries temporais} \label{section1}

A base de séries temporais para a estatística é o modelo de Média Móvel Integrada Autorregressiva (ou em inglês, Autorregressive Integrated Moving Average - ARIMA) cujos valores (que podem ter uma ou mais diferenciações) dependem dos valores passados e dos erros de períodos passados.

\subsection{Passeio aleatório, modelo ARIMA e derivados} \label{subsection11}

Considere um processo do tipo

$$ y_{t} = \epsilon $$

Conhecido como passeio aleatório

O processo

$$ y_{t} = \alpha y_{t-1} + \epsilon $$

\subsection{Critérios de avaliação dos modelos} \label{subsection12}

Normalmente, o ajuste do modelo é feito levando em consideração toda a base de dados disponível ou ainda, pode ser que uma amostra com os dados mais recentes.
Uso de critérios de informação, como AIC, BIC, para quantificar a qualidade do ajuste

\section{Abordagem da ciência de dados} \label{section2}

Lorem ipsum
sit amet

\subsection{Modelos}  \label{subsection21}

Lorem ipsum

\subsection{Critérios de avaliação dos modelos} \label{subsection22}

Lorem ipsum
sit amet

\section{Benchmark} \label{section3}

Lorem ipsum
sit amet

\section{Conclusão}

Lorem ipsum
sit amet

\begin{thebibliography}{99}
  \bibliographystyle{unsrt}

\bibitem{someonecited} A. Someone, \textit{Lorem ipsum
    sit amet vol. 1}, (Consectetur Inc, New York, 1952), 3th ed.

\end{thebibliography}

\end{document}
