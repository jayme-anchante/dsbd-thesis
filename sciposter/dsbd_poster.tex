%-----------------------------------------------------------------------
%                                            Prof. Dr. Walmes M. Zeviani
%                                leg.ufpr.br/~walmes · github.com/walmes
%                                        walmes@ufpr.br · @walmeszeviani
%                      Laboratory of Statistics and Geoinformation (LEG)
%                Department of Statistics · Federal University of Paraná
%                                       2019-fev-15 · Curitiba/PR/Brazil
%-----------------------------------------------------------------------

%-----------------------------------------------------------------------
% Intruções de compilação.

%-----------------------------------------------------------------------

% Classe do documento: http://ctan.math.utah.edu/ctan/tex-archive/macros/latex/contrib/sciposter/scipostermanual.pdf
\documentclass[portrait, 24pt, final]{sciposter}

%\usepackage{fancybullets}
%\usepackage{other packages you may want to use}

%\definecolor{BoxCol}{rgb}{0.9,0.9,0.9}
% uncomment for grey background to \section boxes
% for use with default option boxedsections

%\definecolor{BoxCol}{rgb}{0.9,0.9,1}
% uncomment for light blue background to \section boxes
% for use with default option boxedsections

%\definecolor{SectionCol}{rgb}{0,0,0.5}
% uncomment for dark blue \section text

%-----------------------------------------------------------------------
% Pacotes.

\usepackage[brazil]{babel}
\usepackage[utf8]{inputenc}
\usepackage{graphicx}

\usepackage{amsmath, amsfonts, amssymb, amsxtra, amsthm}
\renewcommand{\labelitemi}{\small $\blacktriangleright$}

\usepackage[mathscr]{eucal}
\usepackage{icomma}
\usepackage{color}
\usepackage{indentfirst}
\setlength{\parindent}{3cm}
\usepackage{multicol}
\usepackage{setspace}
\onehalfspace
\usepackage[hang]{caption}

\usepackage{natbib}
\bibpunct{(}{)}{;}{a}{,}{,}

\usepackage{fancybox}
\usepackage{tikz}
\usetikzlibrary{positioning, shapes, shadows, arrows}

\usepackage{Sweave}

\usepackage{lipsum}

%-----------------------------------------------------------------------

% \definecolor{BoxCol}{rgb}{0.470,0.537,0.568}
\definecolor{BoxCol}{rgb}{0.25, 0.25, 0.25}
% \definecolor{SectionCol}{rgb}{1,1,1}
\definecolor{SectionCol}{rgb}{1,0.5,0.25}

%-----------------------------------------------------------------------
\begin{document}

% Imagem de cabeçalho do poster.
\begin{tikzpicture}[remember picture, overlay]
  \node[anchor = north east, inner sep = 0em]
  % at (current page.north east)
  at (76, 5.5)
  {\includegraphics[width = 1.0\paperwidth]{capa_texto.png}};
\end{tikzpicture}

\conference{
  \normalsize I Encontro de Data Science \& Big Data, 28 de Junho de 2019, Curitiba/PR
}

\vspace{10cm}

\noindent
\begin{minipage}[c][10cm][c]{0.75\textwidth}

  \vspace{6ex}
  {\Huge
    Título do trabalho de conclusão de curso a ser apresentado
    no I Encontro de Data Science \& Big Data
  }

  \vspace{1ex}
  {\Large
    Autor do Trabalho$^1$,
    Professor Orientador$^2$,
    Professor Coorientador$^3$,
    Membro Externo Convidado$^4$
  }

  \vspace{1ex}
  $^1${Aluno do programa de Especialização em Data Science \& Big Data, {\it aluno@dsbd.ufpr.br}};\\
  $^2${Professor do Departamento de Estatística - DEST/UFPR, {\it orientador@dsbd.ufpr.br}};\\
  $^3${Professor do Departamento de Informática - DINF/UFPR, {\it coorientador@dsbd.ufpr.br}};\\
  $^3${Head de Data Science da Empresa XYZ, {\it colaborador@empresa.xyz}};

\end{minipage}

\vspace{3cm}

%-----------------------------------------------------------------------
% Conteúdo do poster.

\begin{multicols}{2}

\begin{abstract}
  \lipsum[1-2]

  \noindent  {\textbf{Palavras chaves:} {
      \it Michaelis Mentem, curvatura intrínseca, regressão, latossolo}.
  }
\end{abstract}

%-----------------------------------------------------------------------
\section*{Introdução}

\begin{itemize}
\item Contextualização do assunto.
\item Abordagens já aplicadas.
\item Casos já conhecidos.
\item Aspectos da problematização.
\item Hipóteses.
\item Justificativa.
\item Objetivos.
\end{itemize}

\section*{Material e Métodos}

\begin{itemize}
\item Sobre o conjunto de dados.
\item Metodologia estatística.
\item Recursos computacionais.
\item Acesso ao material.
\end{itemize}

\section*{Resultados e discussões}

\begin{itemize}
\item Principais resultados.
\item Tabelas.
\item Gráficos.
\end{itemize}

\section*{Conclusões}

\begin{itemize}
\item Conclusões.
\end{itemize}

\section*{Principais Referências}
\begin{spacing}{0.8}
\begin{itemize}
\begin{small}
 \item[] HOTHORN, T.; BRETZ, F.; WESTFALL, P. (2008). Simultaneous Inference in General
         Parametric Models. \textbf{Biometrical Journal}, 50(3), 346--363.
 \item[] R Development Core Team. (2010). R: A language and environment for statistical
         computing. R Foundation for Statistical Computing, Vienna, Austria.
         ISBN 3-900051-07-0, url \textit{http://www.R-project.org.}
\end{small}
\end{itemize}
\end{spacing}

\section*{Agradecimentos}

\noindent
Os autores agradecem á Secretatia de Segurança Pública pela disponibilização dos dados.

\end{multicols}
\end{document}
